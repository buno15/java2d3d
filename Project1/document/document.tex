\documentclass[12pt]{article}
\usepackage[utf8]{inputenc}
\usepackage{graphicx}
\usepackage{hyperref}
\usepackage{geometry}
\geometry{a4paper, margin=1in}

\title{Java 2D\&3D Graphics Project1}
\author{m5271506 Kiyohiro Murai}
\date{2023-12-27}

\begin{document}

\maketitle
\tableofcontents
\newpage

\section{Classes}
Please see code for details. Here it will be written an overview of each class.
\subsection{ContourPlotMain.java}
Class containing the main method. Controls GUI construction, data reading, etc. It is responsible for arranging the contour panels described later in the JFrame and visually displaying it.
\subsection{ContourLinePlotPanel.java}
Panel to draw contour lines. Create a panel that draws contour lines based on the shape data read from the vtk file, the color data from the color map csv file and isoValues.
\subsection{FilledContourPlotPanel.java}
Create a panel to draw filled contour lines. The data also used is vtk and color map csv.
\subsection{Point.java}
A class that handles the point of shapes. This class has x, y, z=0 and scalar values defined as fields. Additionally, it includes methods to normalize a scalar value to correspond to an isoValues and to scale it to fit the JPanel size.
\subsection{Triangle.java}
A triangle class with three Point.java instances.
\subsection{ColorMap.java}
This is a class for handling colors with specific scalar values from the provided color map csv. Since scalar values are specified in the range [0-1] in csv, it includes a method to normalize to correspond to the scalar values of the shape data.
\subsection{VTKReader.java}
This is a class that reads vtk files and obtains graphic data as Java objects. Get all points, triangles, cell types, max-min points to normalize, max-min scalar values to normalize.
\subsection{ColorMapReader.java}
This is a class for reading colormap csv and handling colors with specific scalar values using Java objects.

\section{Contour plotting algorithm}
ContourLinePlotPanel.java has a drawContourLines() method to draw contour lines. Draws contour lines from the triangles, colormap, and isoValues of the shape given in the constructor.
\subsection{Counting the number of intersections between isoValue and triangle}
findIntersectionPoints() method returns the number of intersections between all sides of the triangle and isoValue. The scalar value of the vertex is compared to isoValue to calculate where the contour line passes through the triangle.
\subsection{Drawing contour line}
drawContourLines() method processes all isoValues specified by the user. For all triangles of the shape given from vtk, use findIntersectionPoints() method described above using the isoValue to calculate the intersection points of each side. If the number of intersection points is 2, contour lines are drawn. The color of the contour line is determined by the scalar value of the intersection point and the colormap using ColorMap.getColorFromScalar() method.

\section{Filled contour plotting algorithm}
\subsection{1}

\section{GUI}

\section{Reference}

\end{document}

